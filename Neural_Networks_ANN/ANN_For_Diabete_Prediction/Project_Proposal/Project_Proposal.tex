\documentclass[rnd]{mas_proposal}
% \documentclass[thesis]{mas_proposal}

\usepackage[utf8]{inputenc}
\usepackage{amsmath}
\usepackage{amsfonts}
\usepackage{amssymb}
\usepackage{graphicx}
\usepackage{todonotes}
\usepackage{hyperref}
\hypersetup{
    colorlinks=true,
    linkcolor=blue,
    filecolor=magenta,      
    urlcolor=cyan,
    pdftitle={Overleaf Example},
    pdfpagemode=FullScreen,
    }

\urlstyle{same}


\title{Convolutionnal Neural Network For Diabete predictions}
\author{Bassole Cedric-Francois/Bouchaala Mohamed}
\date{December 2022}

% \thirdpartylogo{path/to/your/image}

\begin{document}

\maketitle

\pagestyle{plain}

\section{Title \& Idea of this project}
\begin{itemize}
    \item\textbf{Diabetes} is a chronic medical condition which is estimated to affect 415 million people in the world.
    \item 5 million deaths a year can be attributed to diabetes-related complications.
    \item Type 2 diabetes can be prevented and reversed if diagnosed early.\\
   \end{itemize}
   \todo[inline]{ \textbf{So, why not using machine learning to predict diabetes in patients using vital 
statistics by using datasets where patients have been diagnosed. By training a neural network,
we can use it to make predictions on new patients.}}


\section{Related Work}

\subsection{Survey of Related Work}
The risk of Type 2 diabetes was predicted using different machine learning algorithms as these algorithms are highly accurate which is very much required in the health profession
\begin{itemize}
    \item \textbf{Random Forest} \url{https://www.sciencedirect.com/science/article/pii/S1877050920308024}
      \item \textbf{Support Vector Machine} \url{https://www.analyticsvidhya.com/blog/2022/01/diabetes-prediction-using-machine-learning/}
      \item \textbf{Linear Regression} \url{https://github.com/sambit221/diabetes-prediction}

    \end{itemize}

\subsection{Limitation and Deficits in the State of the Art}
\begin{itemize}
    \item \textbf{The most important deficit we have notice with all the previous work is about the Precision.}Indeed The most accurate Machine learning algorithm for Diabete prediction rn is the Random Forest (not really High).
    However This accuracy can be improve by Using Neural Net for a more voluptuous dataset
    \item \textbf{This dataset only has eight categories(features) } which might not be enough to truly make accurate predictions
   \item \textbf{A carrefull approach can permit us to say the number of (1) outcomes are thery
low}. so the model can’t predict accuratly this category of persons
7
\item In the previous studies we also noticed The project purpose seems insuffisant and unfinished 
\end{itemize}

\section{Problem Statement}
Then we decided to use an other approach of the problem:
\begin{itemize}
    \item Use a Convolutional neural Network to try to increse the accuracy for more important datasets about diagnoscticed persons
    \item Moreover we will try to build a useful interface to permet real test for new patients
Our main goal is to increase the precison by training the model and make this useful for medical purpose thanks to the user interface
\end{itemize}

\section{Project Plan}

\subsection{Work Packages}
In order to run through this Project, y
ou must have installed \textcolor{blue}{Python},
setup a notebook,and create a  \textcolor{blue}{virtual environnement},and install
  all these packages:


\begin{itemize}
    \item TensorFlow
    \item Keras
    \item Seaborn
    \item pydot
    \item scikit-learn
    \item pandas
    \item Matplotlib
\end{itemize}

For this project purpose we had use 
\textbf{Anaconda },and setup a virtual environment to use a \textbf{jupiter notebook.}
\\
Learn how to how to install the prerequiste using the link below:

\url{https://towardsdatascience.com/virtual-environments-in-anaconda-jupyter}

Moreover we used all the following purpose por different purpose


\begin{itemize}
    \item VS Code
    \item Flutter SDK and Dart Packages to build the web page wze will use as user interface 
    \item  Firebase API
    \item MykTek and laTex Packages
    \item Pearl
    \item LyX
\end{itemize}

\subsection{Project Schedule}
\textcolor{red}{All The document related to this project will are availlable in Github and 
}
\subsubsection*{Dataset}
\begin{itemize}
    \item \url{https://www.kaggle.com/datasets/uciml/pima-indians-diabetes-database}
\end{itemize}

\subsubsection*{Code Jupiter notebook}
\begin{itemize}
    \item \url{https://github.com/ingeniously/Artificial_intelligence_/tree/main/Neural_Networks_ANN/ANN_For_Diabete_Prediction/Python_code}
\end{itemize}

\subsubsection*{Report and Analysis}
\begin{itemize}
    \item \url{https://github.com/ingeniously/Artificial_intelligence_/tree/main/Neural_Networks_ANN/ANN_For_Diabete_Prediction/LaTex_Report}
\end{itemize}

\subsubsection*{Web interface}
\begin{itemize}
    \item \url{}
\end{itemize}


\nocite{*}

\end{document}
